%\documentclass[10pt,a4paper,twocolumn,draft]{article}
\documentclass[10pt,a4paper,draft]{article}
\usepackage[utf8]{inputenc}
\usepackage[ngerman]{babel}
\usepackage{amsmath}
\usepackage{amsfonts}
\usepackage{amssymb}
\usepackage{url}
\usepackage{hyperref}


\begin{document}
\author{Gruppe 2}
\title{Portierung und Parallelisierung mehrerer Filter von GIMP nach GEGL}
\maketitle

\section*{Abstract}

\part{Color-Exchange}

\section{Einleitung}
 
\subsection{Beschreibung des Filters aus Nutzersicht}
Die Beschreibung des Filters in der GIMP-Dokumentation "Dieses Filter ersetzt eine Farbe durch eine andere."\footnote{\url{http://docs.gimp.org/de/plug-in-exchange.html}} fasst die Grundidee des Filters grob zusammen. Tatsächlich ermöglicht das Filter nicht nur exakt eine Farbe durch eine andere zu ersetzen, sondern auch ähnliche Farben durch eine andere zu ersetzen, wobei in diesem Fall die ersetzende Farbe der jeweiligen Abweichung der im Bild vorhandenen von der zu ersetzenden Farbe angepasst wird.
\subsection{Algorithmus} 
%Beschreibung Algorithmus allgemeinsprachlich, 

%Pseudocode, visuell
\subsection{Zielsetzung}
%Portierung von GIMP nach GEGL
%Parallelisierung mittels OpenMP
\section{Portierung}
%Beibehaltung der Farbrepräsentation in RGBA u8
%Beibehaltung bereits in GIMP vorhandener Bug 

\section{Parallelisierung}
\section{Auswertung}

\subsection{Korrektheit}
 
%Bild Eingabe (Duck / matting-global / car-stack?)
%Vergleich Ausgabe bei Eingabe mit gleichen Parametern GIMP – GEGL Bilder
%Ergebnis vom Diff
\subsection{Laufzeit}
 
%X Durchläufe in Diagram abtragen?
%System beschreiben (Hardware, Software, Umgebungsvariablen) !!!
%Boxplots !!!
%Statistische Analyse ?!
%Aus Debug-Output Diagramm erstellen um Anteil des Filters an Gesamtlaufzeit zu verdeutlichen
\section{Evaluation der Ergebnisse}

%Warum wird es mit OpenMP langsamer?
\section{Ausblick}
%Umstellung auf Floats
%Behebung des Bugs
%Implementierung in OpenCL
\end{document}