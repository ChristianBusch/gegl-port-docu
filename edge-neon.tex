%% edge-neon.tex

\subsection{Neon}
\paragraph{Beschreibung des Filters aus Nutzersicht}
\paragraph{Algorithmus} 
%Beschreibung Algorithmus allgemeinsprachlich, 
%Pseudocode, visuell
\paragraph{Portierung}
\paragraph{Parallelisierung}

was gemacht wurde111:
1) Einige unnötige Sprachkonstrukte entfernt
2) Auch viele Variablen umbenannt

Parallelisierbarkeit:
Der Algorithmus sieht folgender Weise aus:
	1) zum Anfang erfolgt waagerechte Suche 
	2) dann Senkrechte,
	3) zum Schluss werden beide Ergebnisse gemergt
	
Es gab auch ursprünglich keine funktionale Abhängigkeit zwischen den beiden Suchstufen (Schritt 1 und 2), 
jedoch eine Datenabhängigkeit: Die Ergebnisse ...
Durch zusätzliche Variablen wurde auch diese Abhängigkeit eliminiert.

Beide Suchstufen erfolgen erfolgen spalten- bzw. zeilenweise und zwar von beiden Seiten simultan,
so ist in diesem Fall unbeliebte automatische Kachelzerlegung sehr vorteilhaft, da sie Wahrscheinlichkeit dafür erhöht, 
dass die Werte von links nach rechts immer noch in Cache bleiben, wenn sie von rechts nach links benötigt werden.
Somit ist 
